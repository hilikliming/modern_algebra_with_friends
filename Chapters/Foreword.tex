\vspace{-0.2in}
\section{Foreword}
% This text takes direct inspiration from the MATH 421 Course that was offered at Miami University in Oxford, OH in the Fall semester of 2012. This course was instructed by the co-author's professor Dr. Katherine Magurn. Dr. Magurn was gifted, unlike any other, in her ability to take dense material, that would otherwise \textit{only} be accessible to mathematicians with some prior knowledge, and make it accessible to individuals who had little or no formal background in mathematics. This course was intended to not only review the fundamental topics of Abstract Algebra, but to teach students how to \textit{generate} and \textit{verify} substantial proof of claims. The world of mathematics is a vast and varied one, but across almost every discipline in mathematics, the tools learned here can again be used to gain deep understanding about newly discovered, defined, or considered mathematical objects and structure. 

% The authors' original intent for this text is to  provide: 1) an overview of the tools and techniques used to make formal arguments and proofs about particular mathematical structure(s) or objects. 2) provide a primer on topics of Modern Algebra that make topics of Category Theory far more accessible to individuals who have not pursued an undergraduate mathematics degree (because let's face it, it is no trivial task to learn Category Theory without \textit{at least} an undergraduate Math degree). If you are reading this text for the second of these purposes, might we suggest the reader consider Steven Roman's text \textit{``An Introduction to the Language of Category Theory"} after finishing this text, as many of the examples covered in Roman's text are drawn from the topics covered in this course. In his book, Roman generalizes many  concepts of Abstract Algebra, covered here, into concepts that pertain to the language of Category Theory.
This text is inspired by the MATH 421 course at Miami University (Oxford, OH) in Fall 2012, taught by Dr. Katherine Magurn. Dr. Magurn excelled at making complex mathematical material accessible to students with little or no formal math background. The course aimed to review key Abstract Algebra topics while teaching students how to generate and verify rigorous mathematical proofs. The concepts learned here apply across various branches of mathematics, providing tools to understand new mathematical structures and objects.

The authors' goal for this text is twofold: (1) to introduce the tools and techniques used to make formal arguments and proofs about mathematical structures, and (2) to serve as a primer on Modern Algebra that prepares readers for Category Theory, which can be challenging without an undergraduate math background. For readers pursuing the latter, we recommend Steven Roman's \textit{An Introduction to the Language of Category Theory} as a follow-up, as it builds on the Abstract Algebra concepts covered in this text and connects them to Category Theory.

This text was originally created as a supplementary set of notes to accompany I. N. Herstein's \textit{Topics in Algebra (2nd ed.)}\nocite{herstein1991topics}, but it has since been expanded into a standalone course, while maintaining the original numbering for theorems, lemmas, corollaries, and other results from Herstein's work.

This text is intended to be accessible to an undergraduate who has completed Algebra II or an eager high schooler who has done the same. We hope that this class will give you the tools required to begin study in a great variety of disciplines of higher mathematics; it will teach you many methods of proof and the fundamental tools of first-order (and second-order) logic while simultaneously teaching you the fundamentals of Modern Algebra: relations, group theory, ring theory, field theory, etc. What's more is that from this very abstract, high-level point of view, the proofs are simple and elegant. 
\noindent The authors believe that understanding this material will unlock many doors for the readers and as such we hope that you will use this newfound knowledge for the good of mankind. We are all alone here on this rock, please take care of one another.\steezybreak
\small
\begin{quote}
Abou Ben Adhem (may his tribe increase!)\\
Awoke one night from a deep dream of peace,\\
And saw, within the moonlight in his room,\\
Making it rich, and like a lily in bloom,\\
An angel writing in a book of gold:—\\
Exceeding peace had made Ben Adhem bold,\\
And to the presence in the room he said,\\
"What writest thou?"—The vision raised its head,\\
And with a look made of all sweet accord,\\
Answered, "The names of those who love the Lord."\\
"And is mine one?" said Abou. "Nay, not so,"\\
Replied the angel. Abou spoke more low,\\
But cheerly still; and said, "I pray thee, then,\\
Write me as one that loves his fellow men."\steezybreak

The angel wrote, and vanished. The next night\\
It came again with a great wakening light,\\
And showed the names whom love of God had blest,\\
And lo! Ben Adhem's name led all the rest.\steezybreak

- J. H. Leigh Hunt
\end{quote} 

