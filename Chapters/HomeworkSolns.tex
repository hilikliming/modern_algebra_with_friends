\renewcommand{\thechapter}{HS}
\chapter*{Homework Solutions}
\addtocounter{chapter}{1} % Manually increment the chapter counter
\markboth{\sffamily\normalsize\bfseries Homework Solutions}{} % Set the chapter header
\addcontentsline{toc}{chapter}{\textcolor{ocre}{Homework Solutions}}
\setcounter{section}{0}
\vspace{-0.3in}
\begin{tcolorbox}
    I encourage you to give each of the homework sets an honest attempt before looking at the solutions, you will get much more out of reading the solutions if you have first spent time carefully thinking about the problems on your own!
\end{tcolorbox}
\vspace{-0.2in}
\section{Homework \#1 Solutions}

\begin{enumerate}
    \item We are given the relation $\sim$ on $\R$:  
          \[
          \forall \ r,s \in \R, \quad r \sim s \iff r - s \in \Z
          \]
          and we are to show that $\sim$ is an equivalence relation.

          \textit{Proof:}  
          \begin{enumerate}
              \item \textbf{Reflexivity:} Take $r \in \R$. Then $r - r = 0$, and $0$ is an integer, so $r \sim r$.
              \item \textbf{Symmetry:} Assume $r \sim s$. Then $r - s$ is an integer.  
              Its ``opposite'' $-(r - s)$ is also an integer, so $s - r \in \Z$ and we have $s \sim r$.
              \item \textbf{Transitivity:} Assume $r \sim s$ and $s \sim t$.  
              Then $r - s \in \Z$ and $s - t \in \Z$. The sum of two integers is again an integer, so  
              \[
              (r - s) + (s - t) \in \Z
              \]
              so $r - t \in \Z$, so $r \sim t$. $\blacksquare$
          \end{enumerate}

    \item \begin{enumerate}
              \item The equivalence class of $\pi$ is given by:
                    \[
                    \text{cl}(\pi) = \{ r \in \R \mid r \sim \pi \}
                    \]
                    which means:
                    \begin{align*}
                        \text{cl}(\pi) &= \{ r \in \R \mid r - \pi \text{ is an integer} \} \\
                        &= \{ r \in \R \mid r = ( \text{an integer} ) + \pi \} \\
                        &= \{ n + \pi \mid n \in \Z \} = \{ \dots, -1+\pi, \pi, 1+\pi, 2+\pi, \dots \}
                    \end{align*}
                    This is usually expressed as $\pi + \Z$.
                    \begin{center}
                        \hspace{-0.1in}\includegraphics[width=0.85\textwidth]{Figures/EquivClassOfPiNumberLine.png}
                    \end{center}

              \item For any $s \in \R$, we can similarly find $\text{cl}(s)$ by substituting $s$ for $\pi$, so:
                    \[
                    \text{cl}(s) = s + \Z.
                    \]
                    Note that $\text{cl}(0)$ and $\text{cl}(1)$ are identical; they are both $\Z$.  
                    However, there are infinitely many $s \in \R$ between $0$ and $1$, and each such $s$ represents a class: $s + \Z$.  
                    So under the equivalence relation $\sim$ in problem (1), $\R$ is broken into infinitely many equivalence classes.
          \end{enumerate}
\end{enumerate}
% Problems 3-6
\begin{enumerate}
    \setcounter{enumi}{2}
    \item The following is an equivalence relation:  
          \[
          \forall n,m \in \Z, \quad n \sim m \iff n + m \text{ is even}.
          \]
          \textit{Proof:}  
          \begin{enumerate}
              \item \textbf{Reflexivity:} Take $n \in \Z$. Then $\underset{\substack{\ \ \ \ \ \ \ \ \ \ \ \ \ \ \ \uparrow \\ \ \ \ \ \ \ \text{the }``k" \text{ from defn. } \\
              \ \ \ \ \ \ \text{of ``even''}}}{n + n = 2n}$, which is even, so $n \sim n$.
              \item \textbf{Symmetry:} Assume $n \sim m$. Then $n + m$ is even, so there exists some $k \in \Z$ such that $n + m = 2k$.  
              Since addition in $\Z$ is commutative (i.e. $n+m=m+n$), we have $m + n = 2k$, which shows $m + n$ is even. Thus, $m \sim n$.
              \item \textbf{Transitivity:} Assume $n \sim m$ and $m \sim p$.  
              Then $\exists \ k \in \Z \ni $ $n + m = 2k$, and $\exists \ l \in \Z \ni $$m + p = 2l$.  
              Now,
              \[
              n + p = (2k - m) + (2l - m) = 2(k - m + l).
              \]
              Since $k, m, l$ are integers, $(k-m+l)$ is an integer, so $n+p$ is even, and $n \sim p$. $\blacksquare$
          \end{enumerate}

    \item Take $n \in \Z$; $n$ is either even or odd.  
          \begin{itemize}
              \item If $n$ is even, then $n + m$ is even $\iff m$ is even.  
              Since any even integer can be used for $m$, we have:
              \[
              \text{cl}(\text{any even } n) = 2\Z, \quad \text{the even integers.}
              \]
              \item If $n$ is odd, then $n + m$ is even $\iff m$ is odd.  
              This makes:
              \[
              \text{cl}(\text{any odd } n) = 1 + 2\Z, \quad \text{the odd integers.}
              \]
          \end{itemize}
          So the relation $\sim$ in (3) breaks $\Z$ into two classes: $2\Z$ and $1 + 2\Z$.

    \item Given $\sigma: \mathbb{R} \to \mathbb{Z}$ where $\sigma(r)$ is the smallest integer greater than or equal to $r$.  
          \[
          \sigma \text{ is not a bijection because it is not injective:}
          \]
          \[
          \frac{1}{2} \neq \frac{3}{4}, \text{ but } \sigma\left(\frac{1}{2}\right) = \sigma\left(\frac{3}{4}\right) = 1.
          \]

    \item Given $\alpha, \beta, \gamma: S \to S$ with $\gamma$ bijective, prove that $\alpha \circ \gamma = \beta \circ \gamma \implies \alpha = \beta$.  \\
          \textit{Proof:}  
          
              Since $\gamma$ is bijective, by L 1.2.3 $\exists$ $\gamma^{-1}: S \to S \ \ni \ \gamma \circ \gamma^{-1} = \gamma^{-1} \circ \gamma = I_S$. \\
              By hypothesis, $\alpha \circ \gamma = \beta \circ \gamma$ ; \ 
              Compose both sides with $\gamma^{-1}$ on the right:
                    \[
                    (\alpha \circ \gamma) \circ \gamma^{-1} = (\beta \circ \gamma) \circ \gamma^{-1}.
                    \]
              By associativity of composition (L 1.2.1) we can shift brackets:
              \begin{align*}
                &\alpha \circ (\gamma \circ \gamma^{-1}) = \beta \circ (\gamma \circ \gamma^{-1}). &\\
                \iff &\alpha \circ I_S = \beta \circ I_S & (\text{since }\gamma \circ \gamma^{-1} = I_S) \\
                \iff &\alpha = \beta. \ \ \ \ \ \blacksquare &
              \end{align*}
                    
              (For the last step, note that $\forall \ s\in S, (\alpha \circ I_S)(s) = \alpha[I_S(s)]=\alpha(s)$, so $\alpha \circ I_s = \alpha$. Likewise, $\beta \circ I_s =\beta$.)
          
\end{enumerate}


\newpage

\begin{tcolorbox}
    I encourage you to give each of the homework sets an honest attempt before looking at the solutions, you will get much more out of reading the solutions if you have first spent time carefully thinking about the problems on your own!
\end{tcolorbox}
\vspace{-0.2in}
\section{Homework \#2 Solutions}