%\chapterimage{chapter_head_1.pdf} % Chapter heading image
\setcounter{chapter}{1}
\chapter*{A Brief Primer on Matrix Multiplication}
\addtocounter{chapter}{1} % Manually increment the chapter counter
\markboth{\sffamily\normalsize\bfseries A Brief Primer on Matrix Multiplication}{} % Set the chapter header
\addcontentsline{toc}{chapter}{\textcolor{ocre}{A Brief Primer on Matrix Multiplication}}

\setcounter{section}{0}
\section{Matrix Multiplication}
\label{sec:MatrixMult}
An $m\times n$ matrix has $m$ rows and $n$ columns. Matrix multiplication is defined when the number of columns in the first matrix matches the number of rows in the second. Given an \( m \times n \) matrix \( A \) and an \( n \times p \) matrix \( B \), their product \( C = AB \) is an \( m \times p \) matrix with entries:
\[
C_{ij} = \sum_{k=1}^{n} A_{ik} B_{kj}.
\]
Essentially, the $ij$th entry of the resultant matrix is the dot product of the $i$th column of the left matrix  with the $j$th row of the right matrix.
\subsection{Examples}

\textbf{1. Two Examples of Multiplication of two \( 2 \times 2 \) matrices:}
\[
\text{Let }A = \begin{bmatrix} 1 & 2 \\ 3 & 4 \end{bmatrix}, \quad 
B = \begin{bmatrix} 5 & 6 \\ 7 & 8 \end{bmatrix}
\]
\[
AB = \begin{bmatrix} 1(5) + 2(7) & 1(6) + 2(8) \\ 3(5) + 4(7) & 3(6) + 4(8) \end{bmatrix}
= \begin{bmatrix} 19 & 22 \\ 43 & 50 \end{bmatrix}.
\]

\[
\text{Let }A = \begin{bmatrix} 0 & 1 \\ -1 & 2 \end{bmatrix}, \quad 
B = \begin{bmatrix} 3 & -1 \\ 2 & 4 \end{bmatrix}
\]
\[
AB = \begin{bmatrix} 0(3) + 1(2) & 0(-1) + 1(4) \\ -1(3) + 2(2) & -1(-1) + 2(4) \end{bmatrix}
= \begin{bmatrix} 2 & 4 \\ 1 & 9 \end{bmatrix}.
\]

\newpage 
\noindent \textbf{2. An Example of Multiplication of two \( 3 \times 3 \) matrices:}

\[
    \text{Let } A = \begin{bmatrix} 2 & -1 & 3 \\ 1 & 0 & -2 \\ 4 & 5 & 6 \end{bmatrix}, \quad
B = \begin{bmatrix} 1 & 2 & 3 \\ 0 & -1 & 4 \\ 5 & 6 & -2 \end{bmatrix}
\]
\[
AB = \begin{bmatrix} 
2(1) + (-1)(0) + 3(5) & 2(2) + (-1)(-1) + 3(6) & 2(3) + (-1)(4) + 3(-2) \\ 
1(1) + 0(0) + (-2)(5) & 1(2) + 0(-1) + (-2)(6) & 1(3) + 0(4) + (-2)(-2) \\ 
4(1) + 5(0) + 6(5) & 4(2) + 5(-1) + 6(6) & 4(3) + 5(4) + 6(-2)
\end{bmatrix}
\]
\[
= \begin{bmatrix} 17 & 23 & -4 \\ -9 & -10 & 7 \\ 34 & 36 & 20 \end{bmatrix}.
\]

\subsection{The Identity Matrix}

The identity matrix \( I_n \) is the \( n \times n \) matrix with ones on the diagonal and zeros elsewhere, for example $I_3$, the $3\times 3$ identity matrix, looks like:
\[
I_3 = \begin{bmatrix} 1 & 0 & 0 \\ 0 & 1 & 0 \\ 0 & 0 & 1 \end{bmatrix}.
\]
For any $m \times n$ matrix \( A \), then \( A I_n = A \) and \( I_m A = A \). If $A$ is square ($n\times n$) we have $AI_n =I_nA = A$.

\subsection{Matrix Multiplication \(\mod n \)}

Matrix multiplication modulo \( n \) follows the same rules as standard matrix multiplication, but each entry in the resulting matrix is reduced modulo \( n \). That is, given matrices \( A \) and \( B \), their product \( C = AB \) satisfies:
\[
C_{ij} = \left( \sum_{k} A_{ik} B_{kj} \right) \mod n.
\]

\noindent \textbf{Example:} Let \( n = 5 \) and consider:
\[
A = \begin{bmatrix} 3 & 4 \\ 2 & 1 \end{bmatrix}, \quad
B = \begin{bmatrix} 1 & 2 \\ 3 & 4 \end{bmatrix}.
\]
Computing the entries as usual:
\[
AB = \begin{bmatrix} 
(3\cdot1 + 4\cdot3) & (3\cdot2 + 4\cdot4) \\ 
(2\cdot1 + 1\cdot3) & (2\cdot2 + 1\cdot4) 
\end{bmatrix} 
= \begin{bmatrix} 15 & 22 \\ 5 & 8 \end{bmatrix}.
\]
Reducing modulo \( 5 \):
\[
AB \equiv \begin{bmatrix} 0 & 2 \\ 0 & 3 \end{bmatrix} \mod 5.
\]