% Hello friend, this doc is still in the works but should eventually contain the homework questions for each of the 12 homework sets, beautifully typeset.

\chapter*{Homework Sets}
\addtocounter{chapter}{1} % Manually increment the chapter counter
\markboth{\sffamily\normalsize\bfseries Homework Sets}{} % Set the chapter header
\addcontentsline{toc}{chapter}{\textcolor{ocre}{Homework Sets}}
\section*{Homework \# 1 }
\markright{\sffamily\normalsize Homework \# 1} % Set the section header
\begin{enumerate}
    \item Define relation $\sim$ on $\R$ as follows:
    \begin{align}
        \forall \ r,s \in \R, \ \ \ r\sim s \iff r-s\in \Z \nonumber
    \end{align}
    Prove that $\sim$ is an equivalence relation on $\R$.
    \item Assume that $\sim$ is the equivalence relation on $\R$ from problem 1.
    \begin{enumerate}[label=\alph*)]
        \item Describe the set that is $cl(\pi)$, the equivalence class of pi.
        \item Into how many equivalence classes does $\sim$ break $\R$?
    \end{enumerate}
    \item Use the definition of ``even" given in class to prove that the following relation $\sim$ on $\Z$ is an equivalence relation:
    \begin{align}
        \forall \ n,m \in \Z, n\sim m \iff n+m \text{ is even.}\nonumber
    \end{align}
    \item Describe the equivalence classes of $\Z$ under the relation $\sim$ in problem 3.
    \item Suppose $\sigma : \R \mapsto \Z$ is the ``ceiling function": $\forall r\in \R,$ $\sigma(r)$ is the smallest integer $\geq r$. Is $\sigma$ a bijection? (Either prove it or give a counterexample).
    \item Assume $\alpha, \beta,$ and $\gamma$ are mappings from a set $S$ to itself, with $\gamma$ both injective and surjective. \\ \\
    Prove: If $\alpha \circ \gamma = \beta \circ \gamma$, then $\alpha = \beta$.
\end{enumerate}
\newpage

%\thispagestyle{plain}
\section*{Homework \# 2}
\markright{\sffamily\normalsize Homework \# 2} % Manually Set the section header
\begin{enumerate}
    \item Assume $a=138,000$ and $b=102,810$. Use the Euclidean Algorithm to calculate the greatest commond divisor of $a$ and $b$.
    \item Same $a$ and $b$ from problem 1. Use matrices to find integers $n$ and $m$ for which
    \begin{align}
        GCD(a,b)=na+mb \nonumber
    \end{align}
    \item Assume $a,b,c\in \Z$, with $a$ relatively prime to both $b$ and $c$. Use Lemma 1.3.1 to prove that $a$ is relatively prime to the product $bc$.
    \item Suppose $n,m,a \in \Z$, with $n$ and $m$ relatively prime. Show that if $n|a$ and $m|a$, then $nm|a$.
    \item Consider $\Z$ under the equivalence relation $\equiv \mod n$ and assume $a\in \Z$. Prove that if $a$ is relatively prime to $n$, then \textit{every} element in $[a]$ is relatively prime to $n$.
    \item Suppose $G$ is a group of order $4$; Say $G=\{e,a,b,c\}$ under an (unspecified) operation, with identity element $e$. Prove that $G$ must be abelian by displaying all possible Cayley tables for $G$.
\end{enumerate}
\newpage

\section*{Homework \# 3}
\markright{\sffamily\normalsize Homework \# 3} % manually set the section header
\begin{enumerate}
    \item Assume that $G$ is a group of order $n$. Prove that the number of $e$'s off the main diagonal in the Cayley table for $G$ must be even.
    \item Use the result from problem 1 to prove this: Every group of even order has at least one element of order $2$.
    \item Define
    \begin{align}
        H&= \{2\times 2 \text{ invertible matrices with entries in }\R\text{ whose columns sum to }1\} \nonumber \\
        &= \big\{\begin{pmatrix}
            r&s\\t&u
        \end{pmatrix}\big | r,s,t,u \in \R, ru-st\neq 0, r+t=1, s+u=1\big\} \nonumber
    \end{align}
    Prove that $H<GL_2(\R)$
    \item Assume that $G$ is a group and suppose that every nontrivial element of $G$ has order $2$. Prove that $G$ is abelian.
    \item Assume that $G$ is a group and suppose that $\forall \ x,y \in G$, $\ \ x^2y^2=(xy)^2$. Prove that $G$ is abelian.
    \item Construct the Cayley table for your personal group, taking care not to use different names for the same element. Keep a minimal$^{(*)}$ list of the relations you discover as you fill in the table. Finally, note the order of your group and whether or not it is abelian. \\ \\ \\ $(*)$ ``minimal" means ``as short as possible," so leave out any relation that can be deduced from the others in your list.
\end{enumerate}
\newpage

\section*{Homework \# 4}
\markright{\sffamily\normalsize Homework \# 4} % manually set the section header
\begin{enumerate}
    \item $G$ group, $m\in \Z$. Prove that if $a\in G$ and $a^m=e$ then $o(a)|m$.
    \item Assume all orders in this problem are finite. Let $G$ be a group, and take $a,b\in G$. \\Prove that $o(ab)=o(ba)$.\\ \\
    (HINT: Say $o(ab)=n\in \Z$, and $o(ba)=m\in \Z$, try to show $n\leq m$ and $m\leq n$.)
    \item Show that every cyclic group must be an abelian group. \\ \\ 
\end{enumerate}
    
\begin{definition}[Normalizer (of a group element)]\hspace{0.1in} \\
The normalizer of $a$ in $G$, denoted $N(a)$, is defined to be the set of group elements which commute with $a$ under the group operation:
\begin{align}
    N(a)=\{g\in G| ga=ag\}. \nonumber
\end{align}
$N(a)$ is also known as the ``centraliser" of $a$ (esp. British usage).
\end{definition}

\begin{enumerate}
\setcounter{enumi}{3}
    \item \begin{enumerate}[label=\alph*)]
        \item Given $a\in G$, prove that $N(a)<G$.
        \item Refer to the handout with the Cayley table for $S_3$ and $\forall a \in S_3$, find $N(a)$.
        \item If $G$ is abelian, what can you say about the normalizers in $G$?
    \end{enumerate}
    \item \begin{enumerate}[label=\alph*)]
        \item In $\Z_{54}$, find subgroups $H=\langle 20 \rangle$ and $K= \langle 18 \rangle$.
        \item Determine $i_G(H)$, $i_H(K)$, and $i_G(K)$
        \item List the right cosets of $K$ in $H$
    \end{enumerate}
    \item Use the Euler $\phi$-function to reduce $41^{1581}\mod 21$
\end{enumerate}
\newpage
\section*{Homework \# 5}
\markright{\sffamily\normalsize Homework \# 5} % manually set the section header


\section*{Homework \# 6}
\markright{\sffamily\normalsize Homework \# 6} % manually set the section header

\section*{Homework \# 7}
\markright{\sffamily\normalsize Homework \# 7} % manually set the section header

\section*{Homework \# 8}
\markright{\sffamily\normalsize Homework \# 8} % manually set the section header

\section*{Homework \# 9}
\markright{\sffamily\normalsize Homework \# 9} % manually set the section header

\section*{Homework \# 10}
\markright{\sffamily\normalsize Homework \# 10} % manually set the section header

\section*{Homework \# 11}
\markright{\sffamily\normalsize Homework \# 11} % manually set the section header

\section*{Homework \# 12}
\markright{\sffamily\normalsize Homework \# 12} % manually set the section header

\vspace{0.35cm}
\textit{If you're feeling generous, feel free to add the \href{https://www.dropbox.com/sh/759pwsjyc3ix0jy/AADBiIq0ZkDsvyEG6fMvSxJta/Abstract\%20Algebra\%20Homework\%20Sets?dl=0&subfolder_nav_tracking=1}{other homework problems} you have worked through so far... I will try to get them all in here eventually!}