\setcounter{chapter}{4}
\chapter*{Course Tests/Exams}
\addtocounter{chapter}{1} % Manually increment the chapter counter
\markboth{\sffamily\normalsize\bfseries Course Exams}{} % Set the chapter header
\addcontentsline{toc}{chapter}{\textcolor{ocre}{Course Tests/Exams}}
\setcounter{section}{0}

\section{Exam \# 1 }
\subsection{Exam \# 1 Take-Home Portion}
\noindent\textbf{Instructions:} Work by yourself. You may use \textit{only} your notes, homeworks, answer sheets, and text from this class. Direct any questions to me, not other students. Answers will be graded on correct mathematical style as well as substance, so make your statements carefully. (5 questions, 50 pts. total)

\begin{definition}[Center of a Group]
The \textit{center} of a group \( G \), denoted \( Z(G) \), is the set of elements in \( G \) which commute with all elements in \( G \). In symbols:
\[
Z(G) = \{ z \in G \mid zg = gz \ \forall \ g \in G \}.
\]
\end{definition}

\begin{definition}[Square (element in a Group)]
An element \( x \) in a group \( G \) is a \textit{square} if there exists \( a \in G \) such that \( x = a^2 \). Denote by \( S \) the set of squares in \( G \):
\[
S = \{ x \in G \mid \exists \ a \in G \ni \  x = a^2 \}.
\]
\end{definition}
\vspace{0.1in}
\begin{enumerate}
    \setlength{\itemsep}{15pt} % Adjusts spacing between items
    \item
    \begin{enumerate}
        \setlength{\itemsep}{10pt} % Adjusts spacing between items
        \item[(a)] Prove that \( Z(G) < t G \). \hfill (4 pts.)
        \item[(b)] Prove that if \( G \) is abelian, then \( S < G \). \hfill (4 pts.)
        \item[(c)] Find \( Z(G) \) and \( S \) for your personal group \( G \). \hfill (4 pts.)
        \item[(d)] Prove that if \( o(G) \) is odd, then \( S = G \). \hfill (4 pts.)
    \end{enumerate} 
    
    \item Assume that \( G \) is a group, and \( x \in G \). \\
    \noindent Prove: If \( x^a = e \) and \( x^b = e \), then \( x^{GCD(a,b)} = e \). \hfill (7 pts.) \\
    (\( a \) and \( b \) are integers, not both zero.)
    
    \item Suppose \( x \) is the only element of order 2 in group \( G \). Prove that \( x \in Z(G) \). \hfill (7 pts.)

    \textbf{Hint:} Take \( y \in G \) and investigate the order of \( y^{-1}xy \).
\end{enumerate}

\newpage
\begin{enumerate}
    \setcounter{enumi}{3} % Continue enumeration from previous section
    \setlength{\itemsep}{15pt} % Adjusts spacing between items
    \item Suppose \( G \) is a group with \( a, b, c \) three different elements in \( G \). If 
    \[
    a^2 = b, \quad b^2 = c, \quad \text{and} \quad c^2 = a,
    \]
    determine \( o(a) \). \hfill (8 pts.)

    \item Assume \( G \) is a group with generators \( a \) and \( b \) and relations
    \[
    \boxed{a^6 = e, \quad b^4 = e, \quad ba = a^5 b.}
    \]
    Assume also that \( G \) has 24 elements, named as follows:
    
    \[
    \boxed{
        \begin{array}{l}
            e, a, a^2, a^3, a^4, a^5,  \\ 
            b, ab, a^2b, a^3b, a^4b, a^5b, \\ 
            b^2, a b^2, a^2 b^2, a^3 b^2, a^4 b^2, a^5 b^2, \\ 
            b^3, a b^3, a^2 b^3, a^3 b^3, a^4 b^3, a^5 b^3.
        \end{array}
        }
    \]

    \begin{enumerate}
        \setlength{\itemsep}{10pt} % Adjusts spacing between items
        \item[(a)] Use the given relations to prove that \( b^2 \) is in the normalizer of \( a^2 \). \hfill (4 pts.)
        
        \item[(b)] Is \( \langle a^2, b^2 \rangle \) cyclic? Support your assertion. \hfill (4 pts.)
        
        \item[(c)] Which element of \( G \) is \( (a^4 b)^{-1} \)? Support your assertion. \hfill (4 pts.)
    \end{enumerate}
\end{enumerate}
\vspace{0.25in}
\begin{tcolorbox}
    \begin{center}
        In-Class Portion of Exam \#1 begins on the next page.
    \end{center}
\end{tcolorbox}
\newpage

\subsection{Exam \# 1 In-Class Portion}
\textbf{Instructions:} Support your answers for maximum credit. No books, notes, help from pals, or calculators allowed. (5 questions, 50 pts. total)
\vspace{0.1in}
\begin{enumerate}
    \setlength{\itemsep}{15pt} % Adjusts spacing between items
    \item Use matrices to find $n,\ m \in \Z \ \ni \ 6480n+5775m = GCD(6480,5775)$ \hfill (10 pts.)
    \newpage
    \item 
\begin{enumerate}
    \setlength{\itemsep}{10pt} % Adjusts spacing between items
    \item[(a)] Given that $\phi$ denotes the Euler-$\phi$ function, find $\phi(18)$. \hfill (5 pts.)
    \vspace{2.25in}
    \item[(b)] Use Euler's corollary to Lagrange's Theorem to reduce $7^{1802}\mod 18$ to an element of $\Z_{18}$. \\ 
    \mbox{} \hfill (5 pts.)
    \vspace{2.25in}
    \item[(c)] Assume $p$ is prime, $p>150$, reduce $130^p \mod p$ to an element of $\Z_p$. (Support your answer) \\ 
    \mbox{} \hfill (5 pts.)
    
\end{enumerate}
\newpage
\item Assume $G$ is cyclic of order $32$, with generator $a$. Suppose $H$ is the subgroup generated by $a^8$, i.e. $H= \langle a^8 \rangle$.
\begin{enumerate}
    \setlength{\itemsep}{10pt} % Adjusts spacing between items
    \item[(a)] List the elements in the coset $Ha^3$. \hfill (6 pts.)
    \vspace{2.5in}
    \item[(b)] Is $Ha^3<G$? (Support your answer.) \hfill (4 pts.)
    \vspace{2.5in}
\end{enumerate}
\item Suppose $G$ is a group and $x\in G, \ x\neq e$. \\
Given $x^8=x^2$, what can you say about $o(x)$? \hfill (5 pts.)
\newpage
\item Assume $G$ is a group with $o(G)=300$. Suppose $G$ has a subgroup $H$ of index $20$ and a subgroup $K$ of order $10$. What are the only two orders possible for the set $HK$? \hfill (10 pts.)
\end{enumerate}
\vfill
\begin{tcolorbox}
    \begin{center}
        This completes Exam \# 1.
    \end{center}
\end{tcolorbox}
\newpage
\section{Exam \# 2 }
\textit{To be continued!}
\subsection{Exam \# 2 Take-Home Portion}
\subsection{Exam \# 2 In-Class Portion}

\section{Exam \# 3 }
\textit{To be continued!}
\subsection{Exam \# 3 Take-Home Portion}
\subsection{Exam \# 3 In-Class Portion}